\section{Requirements}
\begin{enumerate}
\item Users adapt a healthier lifestyle 
	\begin{enumerate}
	\item Functional Requirements
		\begin{enumerate}
		\item Users should sign-up to 
			\begin{enumerate}
			\item Personal information is collected during user sign up
			\item Users that have already signed up are able to login
			\end{enumerate}
		\item User Personal Information Input
			\begin{enumerate}
			\item User inputs static data (Gender, Date of Birth)
			\item User inputs variable data (Height, Weight, Notes)
			\item Variable data is updateable
			\item Variable personal data is kept as history
			\end{enumerate}
		\item User Food Consumption Input
			\begin{enumerate}
			\item User creates menus and re-uses these menus
			\item User selects the group of food item he is searching
			\item A textbox is used to get input from user
			\item USDA database is used to query food that the user inputs
			\item The unit of the selected food is queried from the USDA database
			\item A drop down menu of food items that match user’s search are  presented and user selects from this list
			\item User inputs the amount of given food where the interface prints out the unit
			\item User inputs food consumption data for a given date
			\item User is able to select a date that is different than current date
			\end{enumerate}
		\item User Physical Activity Input
			\begin{enumerate}
			\item http://www.nutristrategy.com/activitylist4.htm​ is used to query physical activity
			\item Physical activity is grouped as daily activity and sports
			\item User inputs the duration of given physical activity
			\item Minutes are always used as unit of physical activity duration
			\end{enumerate}
		\item Insight on User Fitness
			\begin{enumerate}
			\item Software gives insight on calorie comparison
				\begin{itemize}
				\item Total calorie intake from the food user has consumed is known
				\item Calorie burn of user’s physical activity is known
				\item Overall calorie intake and output is compared
				\end{itemize}
			\item Software gives insight on nutrition consumption
				\begin{itemize}
				\item Nutrition intake from the food user has consumed is known
				\item Daily nutrition intake recommendations for user’s profile is known
				\item Overall nutrition intake and recommendation is compared
				\end{itemize}
			\item Software gives insight on Body Mass Index
				\begin{itemize}
				\item Weight and Height of user is known
				\item BMI calculation is done
				\end{itemize}
		\item Nutrition and Calorie Analysis Over Time
			\begin{itemize}
			\item Past data on nutrition input, calorie input and output are known
			\item User selects an interval for analysis
			\item Software provides intake and output comparison over the selected interval
			\end{itemize}
		\end{enumerate}
		\end{enumerate}
\newpage
	\item Non-functional Requirements
		\begin{enumerate}
		\item User continuity
			\begin{enumerate}
			\item The user returns to the application
			\item The user inputs most of his daily activity and food consumption
			\end{enumerate}
		\item User Experience
			\begin{enumerate}
			\item The user easily understands nutrition and calorie analysis provided by the software
			\item The user easily inputs food consumption and daily activity data
			\end{enumerate}
		\item Software Technology
			\begin{enumerate}
			\item The system is developed using Tomcat/Java
			\item MySQL database is used
			\item USDA API is consumed
			\end{enumerate}
		\item Security
			\begin{enumerate}
			\item Unique attribute of a user is his email address
			\item Every email address is associated with a password
			\item Data about a user is only visible if email and password data are present
			\end{enumerate}
		\end{enumerate}
		\item System Requirements
		\begin{enumerate}
		\item System must be python2.7 compliant.
		\item System must operate on ubuntu14.06 32bit or higher distribution.
		\item System must be run with Django 1.10.3, explicitly.
		\item System requires a database with SQL standard compliance.
		\item Network must be publicly available for given network port.		
		\end{enumerate}
	\end{enumerate}
\end{enumerate}
