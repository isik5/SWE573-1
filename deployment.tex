\section{Deployment}
\subsection{Description}
In this section, the steps which must be taken to deploy the application
into production will be explained. To start with, the production environment
must have these dependencies already installed:

\begin{table}[H]
\centering
\caption{My caption}
\label{my-label}
\begin{tabular}{@{}llll@{}}
\toprule
 & Type & Name & Version \\ 
\midrule
1 & OS     & Ubuntu     & 14.04+, 32bit+        \\
2 & Programming Language     & Python    & 2.7       \\
3 & Dependency management & Pip & 8.1.1+ \\
4 & Version control & Git & 1.0+ \\
5 & SSH Agent & OpenSSH & 1.0+ \\
\bottomrule
\end{tabular}
\end{table}


These dependencies are required to be in the production environment,
and their installation processes are not covered in this document. However,
many of these packages are usually included in standard linux distributions.
If Ubuntu 16.04 is chosen as target OS, dependencies 2,3,4,5 will be 
available in the system.

Here are the dependencies of which installation processes are covered in this
document

\begin{table}[H]
\centering
\caption{My caption}
\label{my-label}
\begin{tabular}{@{}llll@{}}
\toprule
 & Type & Name & Version \\ 
\midrule
1 & Web Framework & Django & 1.10.3 \\
2 & Network library & requests & Most recent \\
\bottomrule
\end{tabular}
\end{table}


\subsection{Deployment}

The project is hosted at Github. Please clone the repository via Git:

\begin{lstlisting}
$ git clone https://github.com/onatbas/SWE573
$ cd SWE573
\end{lstlisting}

A dependency.txt file is placed under the repository. This will install
"requests" library for you. Type in this:

\begin{lstlisting}
$ sudo pip freeze > dependencies.txt
\end{lstlisting}

You'll also need to install django manually. To ensure you have the correct
version, please check;

\begin{lstlisting}
$ python -v 
\end{lstlisting}
 gives you 2.7. \\ If Python version is correct, please install django with 
the correct version tag, like this:

\begin{lstlisting}
$ pip install django==1.10.3
\end{lstlisting}

Based on your credentials, you might need to execute this on sudo mode.

\subsubsection{Production}

The current design of the application is based on django's ability to serve
the application. This is done by "runserver" command. Before explaining this,
some setup is required. This is to ensure;
\begin{enumerate}
\item A database is ready to use.
\item An admin account is created.
\end{enumerate}

\subsubsection{Admin Account}
To create the admin account, simply type this in the terminal

\begin{lstlisting}
$ python manage.py createsuperuser
\end{lstlisting}

and type in the required fields.

\subsubsection{Creating Database}
Django creates the database as long as the project has ORM information.
In Nutritrack, these ORM files or migrations are also versioned with the project,
so django has everything to create the tables, schemas etc. For quick SQLite setup,
use this command.

\begin{lstlisting}
$ python manage.py migrate
\end{lstlisting}

\subsubsection{Production}

After previous steps, NutriTrack is ready to go live production. Simply, type 
this in:

\begin{lstlisting}
$ python manage.py runserver 
\end{lstlisting}

If you're using SSH connection to your server instance, you want to be able
to exit the command line while the application is running, to do this, 
screen command may be used.

\begin{lstlisting}
$ screen
// Press enter
$ python manage.py runserver
// CTRL + A + D
\end{lstlisting}

screen is a standard tool that comes with linux dsitributions and is reliable
to execute commands on other terminal instances while another process is running.

You may connect to the same screen instance with this call.
\begin{lstlisting}
$ screen -r
\end{lstlisting}

If the instance has more than one screen instances running at a given time, 
this command will list the screen instances instead of connecting to 
a random one. From the list, id can be read and connection can be made based 
on the id of the screen such as this:

\begin{lstlisting}
$ screen -r 27830
\end{lstlisting}

To kill all the screen instances, simply type this:

\begin{lstlisting}
$ killall screen
\end{lstlisting}










